\subsection{Biblioteca}
Esta é a classe que representa a biblioteca usada pelos clientes para comunicar com o sistema de memória distribuída. Esta classe tem a seguinte estrutura:

\begin{table}[H]
\centering
\begin{tabular}{| p{2cm} | p{5,5cm} |}
\hline
\textbf{Variável} & \textbf{Descrição} \\
\hline
Nº de Servidores & Número total de servidores primários existentes  \\
\hline
TID & Transacção atribuída pelo \textit{Master} \\
\hline
(UID, Servidor) & Lista de associações entre \textit{UID}s e o seu respectivo \textit{Servidor} \\
\hline
Cache de Servidores & Estrutura que mapeia \textit{UID} no \textit{Servidor} em que o \textit{PadInt} está armazenado \\
\hline
Temporizadores & Lista de \textit{timers} para cada servidor \\
\hline
\end{tabular}
\caption{Atributos da classe Biblioteca} \label{lib}
\end{table}

De seguida apresenta-se alguns métodos da Biblioteca:
\begin{itemize}
	\item \textit{bool init()}: A Biblioteca pergunta ao Master qual é o número de servidores primários existentes;
	\item \textit{bool TxBegin()}: A Biblioteca pede ao Master para criar um novo \textit{TID} para a transacção e regista-o;
	\item \textit{PadInt CreatePadInt(int UID)}: A Biblioteca calcula qual o servidor primário onde vai alocar o novo \textit{PadInt}. De seguida pergunta ao Master qual é o endereço do servidor que escolheu e depois de obter o endereço, pede ao servidor para alocar o novo \textit{PadInt}. O servidor primário cria um \textit{PadInt} inicializado a zero, sem \textit{locks} e pede ao secundário para fazer o mesmo, só respondendo à Biblioteca com um \textit{ack} apenas depois de ter recebido o \textit{ack} do secundário. Finalmente a Biblioteca cria o \textit{Stub} do \textit{PadInt} para retornar ao cliente;
	\item \textit{PadInt AccessPadInt(int UID)}: A Biblioteca calcula qual o servidor primário onde vai alocar o novo \textit{PadInt}. De seguida pergunta ao Master qual é o endereço do servidor que escolheu e depois de obter o endereço, pergunta ao servidor se tem o \textit{PadInt}. Caso a resposta seja afirmativa, a Biblioteca retorna ao cliente uma nova instância do \textit{Stub} do \textit{PadInt}. Caso contrário é retornado \textit{null};
	\item \textit{int Read()}: A Biblioteca envia um pedido de leitura para o servidor primário onde está alocado o \textit{PadInt}. O servidor primário invoca o método \textit{obterLockLeitura(TID,UID)} e se obtiver o \textit{lock} de leitura (ou se já possuir o lock de escrita), obtém-se o \textit{PadInt} e o seu valor actual. De seguida o servidor primário envia uma mensagem ao secundário para que execute o método \textit{obterLockLeitura(TID,UID)}, de modo a que os estados fiquem coerentes. Depois do servidor executar o método e enviar um \textit{ack} ao primário o valor do \textit{PadInt} é retornado ao cliente. No caso do \textit{lock} de leitura não for obtido, o servidor primário insere a transacção nos \textit{Leitores à espera} desse \textit{PadInt} sendo os passos anteriores executados mal a transacção saia dos \textit{Leitores à espera} e entre nos \textit{Leitores};
	\item \textit{void Write(int value)}: Este método é em todo semelhante ao /textit{Read()}, sendo alterado o valor actual do \textit{PadInt} em vez de ser lido e o retorno do servidor primário para Biblioteca é um \textit{ack} em vez de um \textit{PadInt}.
\end{itemize}