\section{Solução}

\subsection{2PC vs S2PL vs Timestamps}

Inicialmente considerámos seguir uma abordagem optimista. No entanto, como não sabemos qual é a relação existente entre o número de leituras e escritas, podemos admitir que existe igual número de leituras e escritas. Neste caso o número de conflitos existentes poderá ser grande. 

Considerando que isto seria um potencial bottleneck de performance no nosso sistema, optámos por abandonar as soluções que usam timestamps ou two-phase-commit (2PC). Para além disto outro argumento contra o 2PC é a possibilidade das transacções poderem abortar (se surgir algum conflito) depois de já terem realizado algum trabalho, o que implicaria refazer-se esse trabalho.

Assim, escolhemos utilizar \textit{Strict Two Phase Locking} ou \textit{S2PL}

\subsection{Replicação activa vs replicação passiva}

Depois de escolhermos qual o protocolo a usar, deparámo-nos com a escolha entre replicação activa e passiva. Inicialmente considerámos seguir replicação activa com um protocolo de \textit{Quorum Consensus}. No entanto, por esta precisar de três servidores (em vez dos dois necessários para a replicação passiva) e tendo também em conta que teríamos que enviar e receber mais mensagens do que as necessárias ao usar replicação passiva, optámos por esta última opção.
