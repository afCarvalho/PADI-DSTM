\section{Tolerância a Faltas - Fail, Freeze e Recover}

Sempre que um método do servidor é invocado, é também verificado se o servidor se encontra em estado freeze, fail ou normal.

O servidor primário envia uma mensagem de \textit{i’m alive} ao respectivo servidor secundário a cada x segundos. Caso o secundário não receba a mensagem após o tempo limite, este regista-se no Master como primário e cria uma nova instância de secundário.

Quando o antigo servidor primário, que recebeu o pedido de freeze ou fail e não enviou um \textit{i’m alive} ao respectivo secundário, volta ao estado normal(recebeu recover) verifica se o secundário já assumiu o papel de servidor primário:
\begin{itemize}
\item Se não aconteceu, é enviado um \textit{i’m alive} para o secundário. No caso de ter estado em freeze executa os pedidos que registou e continua o seu funcionamento normal, no caso de ter estado em fail continua a operar de forma normal enquanto primário.
\item Se aconteceu, independentemente do estado anterior ser fail ou freeze, o servidor termina a sua execução.
\end{itemize}
Na situação inversa, ou seja, o servidor secundário não enviou a resposta a um pedido no intervalo de tempo máximo, o servidor primário cria um novo servidor secundário e quando o antigo servidor secundário receber recover, termina a sua execução independentemente do estado anterior. 

No caso em que recebeu recover, vindo do estado de freeze e o servidor primário ainda está a espera da resposta, executa os pedidos pendentes, responde ao servidor primário e continua a operar de forma normal. 

O caso em que o servidor secundário retorna do estado fail, por opção nossa não acontece.A alternativa seria esperar que o servidor voltasse e fazê-lo terminar de seguida, no entanto, como à partida sabemos que se o secundário falhou, assim que o primário atender um pedido o seu estado vai ficar inconsistente, então este termina assim que receber o pedido de fail, parecendo ao servidor primário que não respondeu no intervalo de tempo máximo previsto.

